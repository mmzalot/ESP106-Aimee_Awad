% Options for packages loaded elsewhere
\PassOptionsToPackage{unicode}{hyperref}
\PassOptionsToPackage{hyphens}{url}
%
\documentclass[
]{article}
\title{Markdown\_lab2}
\author{mmz}
\date{1/12/2022}

\usepackage{amsmath,amssymb}
\usepackage{lmodern}
\usepackage{iftex}
\ifPDFTeX
  \usepackage[T1]{fontenc}
  \usepackage[utf8]{inputenc}
  \usepackage{textcomp} % provide euro and other symbols
\else % if luatex or xetex
  \usepackage{unicode-math}
  \defaultfontfeatures{Scale=MatchLowercase}
  \defaultfontfeatures[\rmfamily]{Ligatures=TeX,Scale=1}
\fi
% Use upquote if available, for straight quotes in verbatim environments
\IfFileExists{upquote.sty}{\usepackage{upquote}}{}
\IfFileExists{microtype.sty}{% use microtype if available
  \usepackage[]{microtype}
  \UseMicrotypeSet[protrusion]{basicmath} % disable protrusion for tt fonts
}{}
\makeatletter
\@ifundefined{KOMAClassName}{% if non-KOMA class
  \IfFileExists{parskip.sty}{%
    \usepackage{parskip}
  }{% else
    \setlength{\parindent}{0pt}
    \setlength{\parskip}{6pt plus 2pt minus 1pt}}
}{% if KOMA class
  \KOMAoptions{parskip=half}}
\makeatother
\usepackage{xcolor}
\IfFileExists{xurl.sty}{\usepackage{xurl}}{} % add URL line breaks if available
\IfFileExists{bookmark.sty}{\usepackage{bookmark}}{\usepackage{hyperref}}
\hypersetup{
  pdftitle={Markdown\_lab2},
  pdfauthor={mmz},
  hidelinks,
  pdfcreator={LaTeX via pandoc}}
\urlstyle{same} % disable monospaced font for URLs
\usepackage[margin=1in]{geometry}
\usepackage{graphicx}
\makeatletter
\def\maxwidth{\ifdim\Gin@nat@width>\linewidth\linewidth\else\Gin@nat@width\fi}
\def\maxheight{\ifdim\Gin@nat@height>\textheight\textheight\else\Gin@nat@height\fi}
\makeatother
% Scale images if necessary, so that they will not overflow the page
% margins by default, and it is still possible to overwrite the defaults
% using explicit options in \includegraphics[width, height, ...]{}
\setkeys{Gin}{width=\maxwidth,height=\maxheight,keepaspectratio}
% Set default figure placement to htbp
\makeatletter
\def\fps@figure{htbp}
\makeatother
\setlength{\emergencystretch}{3em} % prevent overfull lines
\providecommand{\tightlist}{%
  \setlength{\itemsep}{0pt}\setlength{\parskip}{0pt}}
\setcounter{secnumdepth}{-\maxdimen} % remove section numbering
\ifLuaTeX
  \usepackage{selnolig}  % disable illegal ligatures
\fi

\begin{document}
\maketitle

\hypertarget{r-markdown-for-lab-2}{%
\section{R Markdown for Lab 2}\label{r-markdown-for-lab-2}}

Install.packages(`knitr')

\hypertarget{chapter-8-functions}{%
\subsection{Chapter 8 (Functions)}\label{chapter-8-functions}}

\hypertarget{write-a-function-f-that-behaves-like-this}{%
\subsubsection{1) Write a function `f' that behaves like
this:}\label{write-a-function-f-that-behaves-like-this}}

'\,'`\{f('Jim')\} '\,'\,'\{``hello Jim, how are you?''
f=function(name)\{ paste(``hello'',name,``how are you?'') \}
f(``Jim'')\}

\hypertarget{write-a-function-sumofsquares-that-behaves-like-this}{%
\subsubsection{2) Write a function `sumOfSquares' that behaves like
this:}\label{write-a-function-sumofsquares-that-behaves-like-this}}

d \textless- c(1,5,2,4,6,2,4,5) sumOfSquares(d)= 21.875

'\,'\,'\{d=c(1,5,2,4,6,2,4,5) mv=mean(d) sumOfSquares=function(y,yy)\{
y=mv - d yy=(y\^{}2) sum(yy) \} sumOfSquares(d)\}

Finially, the method of computation for sum of squares is described:
``To compute the''sum of squares'', subtract the mean value of all
numbers from each number. Square these numbers and sum them'' and also
notes \textsubscript{``(stretch goal: make a variant that can handle NA
values - no extra points, just a challenge)''}

\hypertarget{chapter-10-flow-control}{%
\subsection{Chapter 10 (Flow control)}\label{chapter-10-flow-control}}

To demonstrate if-loops introduced in this chapter, we are asked to sum
sequences according to the following directions:

\hypertarget{write-a-for-loop-that-adds-the-numbers-1-to-10}{%
\subsubsection{3) Write a for loop that adds the numbers 1 to
10}\label{write-a-for-loop-that-adds-the-numbers-1-to-10}}

'\,'\,'\{x=0 for (i in 1:10)\{ x=x+i \} x print(x)\}

The final print command here will yield a final answer of \emph{55}.

\hypertarget{write-a-for-loop-that-adds-the-odd-numbers-between-1-and-10}{%
\subsubsection{4) Write a for loop that adds the odd numbers between 1
and
10}\label{write-a-for-loop-that-adds-the-odd-numbers-between-1-and-10}}

'\,'\,'\{y=0 for (i in 1:10) \{ if (i \%\% 2) y=y+i \} print(y)\}

The final command indicates that the sum of odd numbers between 1 and 10
is equal to \emph{25}.

\end{document}
